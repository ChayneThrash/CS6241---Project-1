%%%%%%%%%%%%%%%%%%%%%%%%%%%%%%%%%%%%%%%%%
% University/School Laboratory Report
% LaTeX Template
% Version 3.1 (25/3/14)
%
% This template has been downloaded from:
% http://www.LaTeXTemplates.com
%
% Original author:
% Linux and Unix Users Group at Virginia Tech Wiki 
% (https://vtluug.org/wiki/Example_LaTeX_chem_lab_report)
%
% License:
% CC BY-NC-SA 3.0 (http://creativecommons.org/licenses/by-nc-sa/3.0/)
%
%%%%%%%%%%%%%%%%%%%%%%%%%%%%%%%%%%%%%%%%%

%----------------------------------------------------------------------------------------
%	PACKAGES AND DOCUMENT CONFIGURATIONS
%----------------------------------------------------------------------------------------

\documentclass{article}
\usepackage{graphicx}
\usepackage{subcaption}
\usepackage{wrapfig}
\graphicspath{ {.} }
\usepackage{listings}
\usepackage{multirow}
\renewcommand\thesubsection{\arabic{subsection}}
\renewcommand\thesubsubsection{\alph{subsubsection}}
\usepackage[justification=centering]{caption}
\usepackage{booktabs}
\usepackage{hyperref}
\hypersetup{
    colorlinks=true,
    linkcolor=blue,
    filecolor=magenta,      
    urlcolor=blue,
}
\usepackage[margin=1.5in]{geometry}
\usepackage{float}
\usepackage{array}
\usepackage{adjustbox}
\newcolumntype{C}[1]{>{\centering\let\newline\\\arraybackslash\hspace{0pt}}m{#1}}


%----------------------------------------------------------------------------------------
%	DOCUMENT INFORMATION
%----------------------------------------------------------------------------------------

\title{CS 6241: Compiler Design \\~\\ Refined Def-Use Analysis Implementation Using Infeasible Paths Information} % Title

\author{Chayne \textsc{Thrash} and Mansour \textsc{Alharthi}} % Author name

\date{\today} % Date for the report

\begin{document}

\maketitle % Insert the title, author and date

\begin{center}
\begin{tabular}{l r}
Instructor: & Professor Santosh \textsc{Pande} % Instructor/supervisor
\end{tabular}
\end{center}

\subsection{Introduction}
In this project, we first implemented an algorithm that detects both the inter- and intra- procedural infeasible paths using branch correlation. Utilizing this information, we implemented a demand-driven def-use analysis that discards the chains along infeasible paths and therefor give a more accurate data flow information. Both algorithms were introduced in the paper "Refining Data Flow Information Using Infeasible Paths" by Gupta, Soffa, and Bodik. Finally, we improved the introduced infeasible path detection algorithm by accounting for more cases than the ones mentioned in the paper. We show our improved version with test results in this report.

\subsection{Improved Infeasible Paths Detection}
We improved the algorithm presented in the paper by adding the ability to handle binary operations with constants to the infeasible path detection algorithm. Specifically, the added operations were addition, subtraction, multiplication, and both signed and unsigned division. This improvement was implemented by adding a stack to the query that kept track of these binary operations. Whenever one of these operations is encountered, both the opcode and the constant involved are placed onto the stack. Whenever the query variable is assigned to a constant, the stack is replayed onto the assigned constant. The resulting value is then used to determine query resolution. 

\subsection{Test Results}
We tested our inter-procedural Def-Use analysis on two benchmarks. The tables below show comparisons between the optimized codes generated using the refined def-use analysis with infeasible paths information and the regular Def-Use analysis. 
~\\~
\begin{adjustbox}{center}
\renewcommand{\arraystretch}{2}
%\begin{table}[]
\begin{tabular}{| C{3cm} | C{3cm} | C{3cm} | C{3cm} | C{3cm} |}
\hline
Benchmark & Def-Use pairs\# without infeasible paths & Def-Use pairs\# with infeasible paths & \% of Def-Use pairs removed \\ \cline{1-4} 
H.264/MPEG-4 AVC & 368 & 189 & 48.6  \\ \cline{1-4}
JPEG-2000 & 368 & 189 & 48.6  \\ \cline{1-4}

\end{tabular}
%\end{table}
\end{adjustbox}
\captionof{table}{Def-Use pairs comparison} 
~\\~
\begin{adjustbox}{center}
\renewcommand{\arraystretch}{2}
%\begin{table}[]
\begin{tabular}{| C{3cm} | C{3cm} | C{3cm} | C{3cm} | C{3cm} |}
\hline
Benchmark & Static size without infeasible paths & Static size with infeasible paths & \% of improvement in static size \\ \cline{1-4} 
H.264/MPEG-4 AVC & 18240 & 18168 & .39  \\ \cline{1-4}
JPEG-2000 & 18240 & 18168 & .39  \\ \cline{1-4}
\end{tabular}
%\end{table}
\end{adjustbox}
\captionof{table}{Static size comparison} 
~\\~
\begin{adjustbox}{center}
\renewcommand{\arraystretch}{2}
%\begin{table}[]
\begin{tabular}{| C{3cm} | C{3cm} | C{3cm} | C{3cm} | C{3cm} |}
\hline
Benchmark & Execution time without infeasible paths & Execution time with infeasible paths & \% of improvement in Execution time \\ \cline{1-4} 
H.264/MPEG-4 AVC & 0.016s & 0.014s & 12.5  \\ \cline{1-4}
JPEG-2000 & 0.006s & 0.005s & 16.7  \\ \cline{1-4}
\end{tabular}
%\end{table}
\end{adjustbox}
\captionof{table}{Execution time comparison} 
\subsection{Work Breakdown}
~\\~
\begin{adjustbox}{center}
\renewcommand{\arraystretch}{2}
%\begin{table}[]
\begin{tabular}{| C{3cm} | C{3cm} | C{3cm} | C{3cm} | C{3cm} |}
\hline
Intra-procedural Infeasible Paths Detection & Intra-procedural Demand-Driven Def-Use Analysis & Inter-procedural Infeasible Paths Detection  & Inter-procedural Demand-Driven Def-Use Analysis & Node \\ \cline{1-5} 
Chayne & Mansour & Chayne & Mansour & Chayne \\ \cline{1-5}

\end{tabular}
%\end{table}
\end{adjustbox}
\captionof{table}{breakdown of work among team members} 

\end{document}
